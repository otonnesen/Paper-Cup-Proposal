%%%%%%%%%%%%%%%%%%%%%%%%%%%%%%%%%%%%%%%%%
% Memo
% LaTeX Template
% Version 1.0 (30/12/13)
%
% This template has been downloaded from:
% http://www.LaTeXTemplates.com
%
% Original author:
% Rob Oakes (http://www.oak-tree.us) with modifications by:
% Vel (vel@latextemplates.com)
%
% License:
% CC BY-NC-SA 3.0 (http://creativecommons.org/licenses/by-nc-sa/3.0/)
%
%%%%%%%%%%%%%%%%%%%%%%%%%%%%%%%%%%%%%%%%%

\documentclass[letterpaper,11pt]{texMemo} % Set the paper size (letterpaper, a4paper, etc) and font size (10pt, 11pt or 12pt)

\usepackage{parskip} % Adds spacing between paragraphs
\setlength{\parindent}{15pt} % Indent paragraphs

%----------------------------------------------------------------------------------------
%	MEMO INFORMATION
%----------------------------------------------------------------------------------------

\memoto{Nadia Ariff, Waste Reduction Manager} % Recipient(s)

\memofrom{Alexander McRae, Oliver Tonnesen} % Sender(s)

\memosubject{The elimination of paper cups} % Memo subject

\memodate{Tuesday, Oct 6, 2018} % Date, set to \today for automatically printing todays date

%\logo{\includegraphics[width=0.3\textwidth]{logo.png}} % Institution logo at the top right of the memo, comment out this line for no logo

%----------------------------------------------------------------------------------------

\begin{document}

\maketitle % Print the memo header information

%----------------------------------------------------------------------------------------
%	MEMO CONTENT
%----------------------------------------------------------------------------------------

\section*{Objective}
This memo provides a proposal for a study to investigae the feasability and
effectiveness of eliminating single-use coffee cups and implementing the
use of shared mugs at all UVic food services
locations.
\section*{Background}
Disposable coffee cup waste is a huge problem at UVic. Despite current programs
to increase landfill diversion rate, ***x\%*** of disposable cups sold on campus
still end up in a landfill.\cite{myrefitem} In addition to these non-diverted
paper cups ending up in landfills and not being properly recycled, the
production and transportation of paper cups and their raw materials has been
shown to emit a large amount of CO$_{2}$.\cite{papercupemissions}
% We should probably not mention carbon emissions, since that's not what
% the RFP was about. I only really put it in so the background didn't look
% so empty. Also, if we do keep it we'll need a better reference than the
% one I grabbed.
\section*{Problem Definition}
\subsection*{Need Statement}
Currently, the vast majority of hot beverage purchases at UVic involve a
disposable paper cup.% Cite some UVic resource with some status supporting this
Of UVic's total non-diverted recyclable waste, ***y\%*** consists of paper
cups. % More stats!
\subsection*{Goals Statement}
Our goal is to transition to near 100\% sustainable/reusable liquid container
use at UVic.
\subsection*{Objectives}
By implementing a rent-a-mug program similar to those already in place at
other academic institutions,% r e f e r e n c e s
we can immensely reduce the number of paper cups sold on campus, and in
turn reduce the number of paper cups sent to landfills.

\section*{Plan of Action}

\section*{Conclusion}

\begin{thebibliography}{9}
\bibitem{myrefitem}
	Who we cited.
	\textit{What we cited}.
	Avaialbe at: Where we cited
	[When we cited].
\bibitem{papercupemissions}
	EPS Distribution. (2018)
	\textit{The Environmental Effect Of Paper Coffee Cups \-- EPS Distribution.}
	Available at:
	http://www.epsdistribution.com.au/the-environmental-effect-of-paper-coffee-cups/
	[Accessed 26 Oct. 2018].

\end{thebibliography}
%----------------------------------------------------------------------------------------

\end{document}
