%%%%%%%%%%%%%%%%%%%%%%%%%%%%%%%%%%%%%%%%%
% Memo
% LaTeX Template
% Version 1.0 (30/12/13)
%
% This template has been downloaded from:
% http://www.LaTeXTemplates.com
%
% Original author:
% Rob Oakes (http://www.oak-tree.us) with modifications by:
% Vel (vel@latextemplates.com)
%
% License:
% CC BY-NC-SA 3.0 (http://creativecommons.org/licenses/by-nc-sa/3.0/)
%
%%%%%%%%%%%%%%%%%%%%%%%%%%%%%%%%%%%%%%%%%

\documentclass[letterpaper,11pt]{texMemo}

\usepackage{parskip}
\setlength{\parindent}{15pt}
\memoto{Nadia Ariff, Waste Reduction Manager}

\memofrom{Alexander McRae, Oliver Tonnesen}

\memosubject{The elimination of paper cups}

\memodate{Tuesday, Oct 6, 2018}

%\logo{\includegraphics[width=0.3\textwidth]{logo.png}} % Institution logo at the top right of the memo, comment out this line for no logo

\begin{document}

\maketitle

\section*{Objective}
This memo provides a proposal for a study to investigae the feasability and
effectiveness of eliminating single-use coffee cups and implementing the
use of shared mugs at all UVic food services
locations.
\section*{Background}
Disposable coffee cup waste is a huge problem at UVic. Despite current programs
to increase landfill diversion rate, ***x\%*** of disposable cups sold on campus
still end up in a landfill.\cite{myrefitem} In addition to these non-diverted
paper cups ending up in landfills and not being properly recycled, the
production and transportation of paper cups and their raw materials has been
shown to emit a large amount of CO$_{2}$.\cite{papercupemissions}
% We should probably not mention carbon emissions, since that's not what
% the RFP was about. I only really put it in so the background didn't look
% so empty. Also, if we do keep it we'll need a better reference than the
% one I grabbed.
\section*{Problem Definition}
\subsection*{Need Statement}
Currently, the vast majority of hot beverage purchases at UVic involve a
disposable paper cup.% Cite some UVic resource with some status supporting this
Of UVic's total non-diverted recyclable waste, ***y\%*** consists of paper
cups.% More stats!
\subsection*{Goals Statement}
Our goal is to transition to near 100\% sustainable/reusable liquid container
use at UVic.
\subsection*{Objectives}
By implementing a rent-a-mug program similar to those already in place at
other academic institutions,% r e f e r e n c e s
we can immensely reduce the number of paper cups sold on campus, and in
turn reduce the number of paper cups sent to landfills.
\subsection*{Constraints}
Our proposal has a fairly large up-front cost; however, we suspect the initial
price to purchase the reusable mugs will be heavily outweighed by the savings
brought by reducing the amount of paper cups that UVic Food Services has to
purchase continuously. Mugs will undoubtedly become lost or broken, and will so
need to be periodically replaced. We expect this cost to also be significantly
lower than purchasing paper cups in the long run. Customers of UVic Food
Services may be upset that they will have to change their routines due to doing
away with paper cups.

\section*{Plan of Action}
The research plan provided below gives a brief overview of some of the questins
we aim to investigate in regards to the implementation of reusable mugs accross
campus.\\\\

\begin{itemize}
	\item Who uses the most paper cups?
	\item What areas of campus are responsible for the largest amount of
		non-diverted waste?\\

	\item When do people most often drink coffee?
	\item How many mugs will be required during peak hours?
	\begin{itemize}
		\item For how long do people use mugs? Do they bring them off campus?
	\end{itemize}
	\item Are they willing to cooperate with UVic's program, or will they
		get their coffee elsewhere?\\

	\item Is it worth implementing this program all over campus, or would it
		make more sense to initially only target residences?
	\item Is this program any better than those already implemented at UVic?
	\item Investigate other institutions that have implemented similar
		programs.\\

	\item Economic considerations:
	\begin{itemize}
		\item Cost/benefit analysis
		\item Opportunity cost
		\item Short term/long term effects
	\end{itemize}
	\item Environmental considerations:
	\begin{itemize}
		\item How much waste can be eliminated by this program?
		\item How does the manufacturing/transport of the mugs affect the
			environment? Is it better or worse than that of paper cups?
	\end{itemize}

\end{itemize}
\section*{Qualifications}
TODO
\section*{Conclusion}
Disposable coffee cup waste is one of the largest areas of potential
improvement with respect to waste diversion.  Implementing a rent-a-mug
program has the potential to help UVic to reach its goal of 75\% waste diversion
by 2021.
\newline
\newline
Our proposed study would determine the feasibility of implementing such a
program. It would investigate the benefits, drawbacks, and the overall cost,
both in the sort- and in the long-term. Undertaking this study would not cost
much to UVic, but could potentially lead to greatly reduced waste production
all over its campus.

\begin{thebibliography}{9}
\bibitem{myrefitem}
	Who we cited.
	\textit{What we cited}.
	Avaialbe at: Where we cited
	[When we cited].
\bibitem{papercupemissions}
	EPS Distribution. (2018)
	\textit{The Environmental Effect Of Paper Coffee Cups \-- EPS Distribution.}
	Available at:
	http://www.epsdistribution.com.au/the-environmental-effect-of-paper-coffee-cups/
	[Accessed 26 Oct. 2018].

\end{thebibliography}

\end{document}
