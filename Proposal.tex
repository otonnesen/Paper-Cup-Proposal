%%%%%%%%%%%%%%%%%%%%%%%%%%%%%%%%%%%%%%%%%
% Memo
% LaTeX Template
% Version 1.0 (30/12/13)
%
% This template has been downloaded from:
% http://www.LaTeXTemplates.com
%
% Original author:
% Rob Oakes (http://www.oak-tree.us) with modifications by:
% Vel (vel@latextemplates.com)
%
% License:
% CC BY-NC-SA 3.0 (http://creativecommons.org/licenses/by-nc-sa/3.0/)
%
%%%%%%%%%%%%%%%%%%%%%%%%%%%%%%%%%%%%%%%%%

\documentclass[letterpaper,11pt]{texMemo} % Set the paper size (letterpaper, a4paper, etc) and font size (10pt, 11pt or 12pt)

\usepackage{parskip} % Adds spacing between paragraphs
\setlength{\parindent}{15pt} % Indent paragraphs

%----------------------------------------------------------------------------------------
%	MEMO INFORMATION
%----------------------------------------------------------------------------------------

\memoto{Nadia Ariff} % Recipient(s)

\memofrom{Alexander McRae, Oliver Tonnesen} % Sender(s)

\memosubject{The elimination of paper cups} % Memo subject

\memodate{Tuesday, Oct 6, 2018} % Date, set to \today for automatically printing todays date

%\logo{\includegraphics[width=0.3\textwidth]{logo.png}} % Institution logo at the top right of the memo, comment out this line for no logo

%----------------------------------------------------------------------------------------

\begin{document}

\maketitle % Print the memo header information

%----------------------------------------------------------------------------------------
%	MEMO CONTENT
%----------------------------------------------------------------------------------------

\section*{Purpose}
Coffee cup waste is a huge problem at UVic. With no clear solution with keeping the coffee cups, we
propose a study that looks at the feasability and effectiveness of completely getting rid of single
use coffee cups and using shared mugs. This will help the University of Victoria reach 75\% waste 
diversion goal and accurately be able to quantify the effect the program has. Also need for other 
paper cup projects will be unneeded potentially saving the University of Victoria millions.
\section*{Purpose & Context (Client Background)}
Despite current programs to increase landfill diversion rate, ***x\%*** of
disposable cups sold on campus at UVic are not diverted, and end up at a landfill.
Our proposal aims to transition to 100\% sustainable liquid container use at UVic.
We plan to achieve this goal by introducing a new rent-a-mug program similar to
those implemented at UBC and Stanford University. A potential constraint of
our proposal is the increase in tuition for all students to initially purchase
and cover replacement of the mugs.

\section*{Plan of Action}

\section*{Conclusion}

%----------------------------------------------------------------------------------------

\end{document}
